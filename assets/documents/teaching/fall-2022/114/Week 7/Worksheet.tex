
\documentclass[11pt]{article}

% Set the margins to be something normal.
\usepackage[margin=0.5in]{geometry}

% Fun lists!
\usepackage{enumitem}
\usepackage{textcomp}
\setitemize{label=\textrightarrow, itemsep=0pt}

% Math symbols.
\usepackage{amsmath, amssymb, amsfonts}
\usepackage{tikz, graphicx, changepage}
\usetikzlibrary{angles, quotes}

% No indents.
\setlength\parindent{0pt}

% Nice fonts :)
\usepackage{tgschola}

\begin{document}
	\thispagestyle{empty}
	{
		\centering
		\huge{Week 7 Recitation Problems} \\
		\Large{MATH:114, Recitations 309 and 310} \\
	}
	\vspace{3em}
	
	1. Determine the lengths of the missing sides in triangles $A$, $B$, and $C$. You don't need any numbers, just variables!
		\vspace{2em}
		\begin{figure}[h!]
			\scalebox{0.9}{
				\begin{tikzpicture}
					\draw (0,0) node[anchor=north]{}
						-- (4, 0) node{}
						-- (4, 3) node{}
						-- cycle;
						
					\node at (2,-1/4) {$a$};
					\node at (4.25, 1.5) {$x$};
					\node at (0, 1.5) {\rule{2cm}{1pt}};
					\node at (2.5, 1) {$A$};
					\node at (0.75, 0.25) {$\theta$};
					\draw (3.7,0)|-(4,0.3);
					
					% Triangle 2
					\draw (6,0) node[anchor=north]{}
						-- (10, 0) node{}
						-- (10, 3) node{}
						-- cycle;
						
					\node at (8,-3/4) {\rule{2cm}{1pt}};
					\node at (10.25, 1.5) {$x$};
					\node at (7.5, 1.5) {$a$};
					\node at (8.5, 1) {$B$};
					\node at (6.75, 0.25) {$\theta$};
					\draw (9.7,0)|-(10,0.3);
					
					% Triangle 3
					\draw (12,0) node[anchor=north]{}
						-- (16, 0) node{}
						-- (16, 3) node{}
						-- cycle;
						
					\node at (14,-1/4) {$a$};
					\node at (17.25, 1.5) {\rule{2cm}{1pt}};
					\node at (13.5, 1.5) {$x$};
					\node at (14.5, 1) {$C$};
					\node at (12.75, 0.25) {$\theta$};
					\draw (15.7,0)|-(16,0.3);
				\end{tikzpicture}
			}
		\end{figure}
		
	\vspace{3em}
	2. Trigonometric functions define relationships between angles and side lengths. Given an angle $\theta$, $$\sin \theta = \frac{\text{opposite}}{\text{hypotenuse}} \hspace{3em} \sec \theta = \frac{\text{hypotenuse}}{\text{hypotenuse}} \hspace{3em} \tan \theta = \frac{\text{opposite}}{\text{adjacent}}$$
	For each of the triangles $A$, $B$, and $C$, express $x$ in terms of a trigonometric function.
	
	\vspace{0.2\paperheight}
	3. For each of the triangles $A$, $B$, and $C$, express the length of the missing side using the answers you found in Problem 2. \textit{(Hint: remember your trig identities!)}
	
	\newpage
	4. Use one of the expressions you found in Problem 3 to set up \textbf{but not solve} the integral $$ \int \frac{1}{\sqrt{4+x^2}} dx.$$ You can use the triangle below for reference.
	
		\vspace{5em}
	
		\begin{figure}[h!]
			\scalebox{0.9}{
				\begin{tikzpicture}
					\draw (0,0) node[anchor=north]{}
						-- (4, 0) node{}
						-- (4, 3) node{}
						-- cycle;
						
					\node at (2,-1/2) {$a = $ \rule{1cm}{1pt}};
					\node at (4.25, 1.5) {$x$};
					\node at (0, 1.5) {\rule{2cm}{1pt}};
					% \node at (2.5, 1) {$A$};
					\node at (0.75, 0.25) {$\theta$};
					\draw (3.7,0)|-(4,0.3);
				\end{tikzpicture}
			}
		\end{figure}
		\vspace{-5em}
	
	\vspace{0.2\paperheight}
	5. Solve $$ \int \frac{1}{\sqrt{25x^2-4}} dx $$ using the fact that $$ \int \sec \theta d\theta = \ln | \sec \theta + \tan \theta | + C. $$
	
		\vspace{5em}
	
		\begin{figure}[h!]
			\scalebox{0.9}{
				\begin{tikzpicture}
					\draw (12,0) node[anchor=north]{}
						-- (16, 0) node{}
						-- (16, 3) node{}
						-- cycle;
						
					\node at (14,-1/2) {$a=$ \rule{1cm}{1pt}};
					\node at (17.25, 1.5) {\rule{2cm}{1pt}};
					\node at (13.5, 1.5) {\rule{10pt}{1pt} $x$};
					% \node at (14.5, 1) {$C$};
					\node at (12.75, 0.25) {$\theta$};
					\draw (15.7,0)|-(16,0.3);t
				\end{tikzpicture}
			}
		\end{figure}
	
\end{document}
