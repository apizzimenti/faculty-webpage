
\documentclass[11pt]{article}

% Set the margins to be something normal.
\usepackage[margin=1in]{geometry}

% Fun lists!
\usepackage{enumitem}
\usepackage{textcomp}
\setitemize{label=\textrightarrow, itemsep=0pt}

% Math symbols.
\usepackage{amsmath, amssymb, amsfonts, nicefrac}

\newcommand{\pp}[1]{#1^{\prime}}

% No indents.
\setlength\parindent{0pt}
\setlength\parskip{1em}

% Nice fonts :)
\usepackage{tgschola}

\begin{document}
	\thispagestyle{empty}
	{
		\centering
		\huge{Week 3 Recitation Problems} \\
		\Large{MATH:114, Recitations 309 and 310} \\
	}
	\vspace{3em}
	1. Let $$f(x) = \frac{1}{2x-1}.$$ Compute the surface area of the solid generated when $f$ is rotated around the $x$ axis where $x$ is between $\nicefrac 34$ and $4$.
	
	\textbf{Solution}: Start by taking the first derivative of $f$:
	\begin{align*}
		\pp f(x) &= \frac{d}{dx}\left(\frac{1}{2x-1}\right) \\
		&= \frac{-1}{(2x-1)^2}
	\end{align*}
	
	Then, we can use the surface area formula:
	\begin{align*}
		S &= \int_{-\frac 34}^4 2\pi f(x) \sqrt{1 + (\pp f(x))^2} \\
		&= 2\pi \int_{-\frac 34}^4 \frac{1}{2x-1} \sqrt{1 + \left(\frac{-1}{(2x-1)^2}\right)^2} \\
		&= 2\pi \int_{-\frac 34}^4 \frac{1}{2x-1} \cdot \int_{-\frac 34}^4 \sqrt{1^2 + \left(\frac{-1}{(2x-1)^2}\right)^2} \\
		&= 2\pi \int_{-\frac 34}^4 \frac{1}{2x-1} \cdot \int_{-\frac 34}^4 \sqrt{\left(1 + \frac{-1}{(2x-1)^2}\right)^2} \\
		&= 2\pi \int_{-\frac 34}^4 \frac{1}{2x-1} \cdot \int_{-\frac 34}^4\left(1 + \frac{-1}{(2x-1)^2}\right) \\
		&= 2\pi \cdot \ln(2x-1) \cdot \left. \left(x + \frac{1}{2(2x-1)}\right)\right\rvert^4_{\nicefrac 34} \\
		&= \frac{221 \ln(\pi)}{4}
	\end{align*}
	so we have found the surface area of our solid.
	
	\newpage
	2. Plot the functions $$ f(x) =x^3, \ g(x) = \sqrt[3]{x}.$$ Rotate the area between $f$ and $g$ around the $x$ axis to form a solid of rotation. Set up (but do not compute) two integrals to find the volume of the solid.
	
	\vspace{15em}
	
	3. Using $f$ and $g$ from \#2, set up (but do not compute) an integral to find the surface area of the solid. Remember that the expression used to find the surface area of a solid is $$S = \int_a^b 2\pi \cdot h(x) \cdot \sqrt{1 + (\pp h(x)^2)}\ dx.$$ How does this integral compare to the integral you set up to compute the volume using the \textit{shell} method? Come up with a geometric explanation (a picture counts!).

	
	\vspace{15em}
	
	4. Let $$f(x) = \frac 12 x^2 - \frac14 \ln (x),$$ and find the length of the curve for $2 \leq x \leq 4$.
	
\end{document}
